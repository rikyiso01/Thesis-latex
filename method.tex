\section{Architecture}\label{architecture}

We have decided to use an architecture similar to the YuraScanner's one:

\begin{figure}
\centering
\includesvg[width=15cm,height=\textheight,keepaspectratio]{figs/.inline-figs/pandocplot9034245466538570907.svg}
\caption{High level architecture}
\end{figure}

Everything is dockerized to allow easy deployment. A dockerized Android
Emulator is used as environment for running the applications. An init
container is used for downloading the application before starting.

\section{Task Executor}\label{task-executor}

The first component we have decided to implement is the task executor.
We have forked and refactored AppAgent in order to satisfy our
requirements as task executor.

\subsection{Structured output}\label{structured-output}

The bridge component of the original AppAgent parses the textual output
of the model to decide the action to pass to the actuator. This causes
errors when the models allucinates due to the non-structured nature of
the response. To solve this we have decided to use the structured output
support of many models. We have modeled the expected response in a JSON
schema and use it to model the response in order to avoid parsing
errors.

\subsection{New actions}\label{new-actions}

New actions have been added to our fork of AppAgent to support a broader
range of tasks

\subsubsection{Registration and login
support}\label{registration-and-login-support}

Action to insert a Hard-coded email and password to remove the problem
of the model that forgets previously used credentials

\subsubsection{Mail one-time-token
support}\label{mail-one-time-token-support}

Action to wait for a one-time-token received on an external mail server
which is accessed using IMAP. The token is delivered via an MQTT broker
by a sidecar container which listens for incoming emails. Since some
verification email only have a clickable link and no code, a headless
Chromium instance is used to open every link present in the email.

For extracting the token from the email, the body of the email is passed
to an LLM. Since some email have a HTML body, to simplify the job of the
LLM, the following pipeline is used to remove useless elements from the
body: - weasyprint: to render the html to a pdf removing invisible
elements - pdftotext: to remove styling elements - llm: to parse the
resulting text

\subsubsection{SMS one-time-token
support}\label{sms-one-time-token-support}

Action to wait for a one-time-token received via SMS. The token is
delivered as for the mail one-time-token via a MQTT broker by a sidecar
container. The container connects to a remote API which lists all the
messages received by a phone number used only for these experiments.

\subsubsection{Wait action}\label{wait-action}

Add a wait action which is used to allow the model to skip making an
action to wait for some action to finish in the application like a
download

\section{Screen detection}\label{screen-detection}

For detecting new states of the application we use the xml dump of the
current layout with some elements removed.

\section{Looping}\label{looping}

Instead of generating the tasks starting from the home page, we generate
and update the list of tasks every time we detect a new state

\subsection{Model selection}\label{model-selection}

AppAgent by default supports only OpenAi. In our fork we have added
support for different LLM providers by using LiteLLM. This allows us to
compare and select a more efficient and less expensive model.

\section{Task extractor}\label{task-extractor}

For each new screen a set of new tasks is generated starting from the
screenshot

\section{Navigator controller}\label{navigator-controller}

The controller is the component that joins the task extraction and the
task executor. Its job is to select the next task to execute and prepare
the environment for the task executor The exploration is finished when a
certain threshold of completed tasks has been reached

\subsection{Navigator graph}\label{navigator-graph}

For deciding the next action and collect results a model of the
application is built which we call Navigation Graph. The Navigation
Graph is a multi-directed graph due to the possibility of different
actions to reach the same screen.

\subsubsection{Nodes}\label{nodes}

A node in the navigation graph is a state of the application which in a
native android application corresponds to the pair activity name and xml
dump of the current layout. To dump the current layout of the
application we used at first ADB dump which produces an XML. Due to its
slowness (2.6 seconds average per call) we decided to switch to
uiautomator2 which is a program that interacts with the automation api
of the device to allow taking dump and screenshot without every time
spawning a new process from scratch.

Each node has a list of possible tasks associated with it, this allows
the task executor to start in the correct state, to avoid starting in a
state where it is impossible to complete the task

\paragraph{Nodes similarity}\label{nodes-similarity}

It is interesting to detect similar nodes, for this we have decided to
use a locality sensitive hash which allows to easily compare the layout
of different nodes to detect similarities. As locality sensitive hash we
have decided to use tlsh.

\subsubsection{Edges}\label{edges}

An edge of the navigation graph is a directed edge which connects two
states of the application. The edge has as label the actions executed to
reach the new state.

\paragraph{Edge stability}\label{edge-stability}

Some applications have elements which depend not only on internal states
but also external factors. Due to the same action in the same state can
cause a different new state. To mark this in the navigation graph, an
edge can be marked as stable or unstable. An edge is marked as stable if
all other edges starting from the same node have a different sequence of
actions otherwise it is marked as unstable.

\subsubsection{Graph compression}\label{graph-compression}

It is possible to obtain a compressed representation of the graph by
condensing together similar nodes into one. This causes previously
unstable edges to become stable.

\paragraph{Compression algorithms}\label{compression-algorithms}

For compressing the nodes together we have decided to hdbscan.

\subsection{Action selection
algorithm}\label{action-selection-algorithm}

For selecting the next task to execute we have defined different
selection strategies:

\subsubsection{Depth first}\label{depth-first}

We select a task from the current node. If the current node is not in
the target application, a previous node in the application is used.

\subsubsection{Breadth first}\label{breadth-first}

We select a task from a node which has available tasks starting from the
entry point of the application.

\subsubsection{Random}\label{random}

A random node is selected

\subsubsection{Random compressed}\label{random-compressed}

Similar to the random algorithm but hdbscan is used to compress similar
nodes

\subsubsection{Weight}\label{weight}

The following formula is used for deciding the next node to select a
task from:

\(execution\_weight(e)=\frac{C_e}{F_e}\)

\begin{itemize}
\tightlist
\item
  \(C_e\): Number of widget without any interaction on them
\item
  \(F_e\): Number of times a node has been passed over
\end{itemize}

\section{Exploration scoring}\label{exploration-scoring}

To evaluate our results in exploring an application we have decided to
define a black box metric for the exploration. Our metric is composed of
two components: - percentage of explored activities: - the total number
of activities is obtained by parsing the Android Manifest inside the APK
- the current activity is obtained using adb - sum of maximum difference
of tlsh of states in each activity: 1. we divide the nodes by activity,
and we compare each node in an activity with each other to obtain the
maximum tlsh difference 2. then we sum all the maximum differences.
